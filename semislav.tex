\documentclass[twocolumn]{book}

\usepackage{xskak}
\usepackage{xcolor}
\usepackage[framemethod=tikz]{mdframed}

\newcommand\mymainline{\mainline[style=styleC]}
\newcommand\myvariation{\variation[style=styleB]}

\newcounter{Lemma}[section]
\newenvironment{Lemma}[1][]{%
  \stepcounter{Lemma}%
  \ifstrempty{#1}%
  {\mdfsetup{%
    frametitle={%
      \tikz[baseline=(current bounding box.east),outer sep=1pt]
      \node[rounded corners=3pt,anchor=east,line width=1pt,draw=black,rectangle,fill=red!60]
    {\strut \color{black}{NN-NN}~\theLemma};}}
  }%
  {\mdfsetup{%
    frametitle={%
      \tikz[baseline=(current bounding box.east),outer sep=1pt]
      \node[rounded corners=3pt,anchor=east,line width=1pt,draw=black,rectangle,fill=red!60]
    {\strut \color{black}{#1}};}}%
  }%
  \mdfsetup{innertopmargin=10pt,linecolor=black,%
            linewidth=1pt,topline=true,%
            frametitleaboveskip=\dimexpr-\ht\strutbox\relax,backgroundcolor=red!10}
  \begin{mdframed}[]\relax%
  }{\end{mdframed}}

\title{Semi-Slav: a guide}
\author{Julien}

\setlength{\columnwidth}{12cm}

\begin{document}

\maketitle

\chapter{Example}

\newchessgame[id=introduction]
\mymainline{1.d4 d5 2. c4 c6 3.Nc3 e6 4.Nf3}

From this point, \myvariation{3.e3 Nf6} is a good but far less aggressive alternative.

\showboard

\begin{Lemma}[Kramnik-Anand (2007)]
\newchessgame[id=firstexample]
\mymainline{1.d4 d5 2. c4 c6 3.Nc3 e6 4.Nf3}
From this point, \myvariation{3.e3 Nf6} is a good but far less aggressive alternative.
\mymainline{4... Nf6 5.e3}
\showboard
\end{Lemma}

\end{document}
